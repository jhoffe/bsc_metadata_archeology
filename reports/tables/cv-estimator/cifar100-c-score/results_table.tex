\begin{tabular}{lrr}
\toprule
Classifier & Gen Score & Test Score \\
\midrule
KNN & 0.798 & 0.816 \\
SVM & 0.843 & 0.848 \\
Naive Bayes & 0.791 & 0.821 \\
Decision Tree & 0.756 & 0.775 \\
Logistic Regression & 0.820 & 0.817 \\
Random Forest & 0.845 & 0.855 \\
AdaBoost & 0.389 & 0.388 \\
MLP & 0.818 & 0.813 \\
XGBoost & 0.842 & 0.854 \\
XGBoost RF & 0.827 & 0.838 \\
\bottomrule
\end{tabular}
